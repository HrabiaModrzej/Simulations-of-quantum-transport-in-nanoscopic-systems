\documentclass[12pt,a4]{article}
%\usepackage{polski}
\usepackage{fontawesome}
\usepackage[utf8]{inputenc}
%\usepackage{tikz}
\usepackage[table]{xcolor}
\usepackage{enumerate}
\usepackage{graphicx}
\usepackage{float}
\usepackage{subcaption}
\usepackage{array}
\usepackage{amsmath}
\usepackage{amssymb}
\usepackage{mathtools}
\usepackage{tabularx}
\usepackage{physics}
\usepackage{booktabs}
\usepackage{multirow}
\usepackage{multicol}
\usepackage{hyperref}
\usepackage{dsfont}
\usepackage{parskip}
\usepackage{tcolorbox}

\usepackage{geometry}

\newgeometry{tmargin=2cm, bmargin=1.5cm, lmargin=2cm, rmargin=2cm}

\usepackage{sectsty}
\definecolor{ColorOne}{RGB}{53, 78, 140}
\definecolor{AGHgreen}{RGB}{0, 105, 60}
\definecolor{AGHblack}{RGB}{30, 30, 30}
\definecolor{AGHred}{RGB}{167, 25, 48}
\sectionfont{\color{AGHgreen}} 
\subsectionfont{\color{AGHgreen}} 
\subsubsectionfont{\color{AGHgreen}} 
\hypersetup{
	colorlinks=true,
	linkcolor=AGHgreen,
	filecolor=AGHgreen,
	urlcolor=AGHgreen,
	citecolor=AGHgreen,
}


%##########################################################################################
%               MOJE KOMENDY
%##########################################################################################

\newcommand{\ods}{\hspace{6mm}}
\newcommand{\noaka}{\noindent}
\newcommand{\aka}{\hspace{6mm}}
\newcommand{\R}{\mathbb{R}}
\newcommand{\krecha}{\rule{\textwidth}{0.25mm}}
\newcommand{\pol}{\frac{1}{2}}


\newcommand{\stopni}{$^\circ$}

\renewcommand{\arraystretch}{1.2}

%\renewcommand{\toprule}{\hline}
%\renewcommand{\bottomrule}{\hline}
%\renewcommand{\midrule}{\hline}


\graphicspath{{./galeria/}}



\newcommand{\machine}[1]{\textcolor{ColorOne}{\texttt{#1}}}

% draw a frame around given text
\newcommand{\framedtext}[1]{%
\par%
\noindent\fcolorbox{black}{gray!7}{%
    \parbox{\dimexpr\linewidth-2\fboxsep-2\fboxrule}{
    \centering
    \parbox{\dimexpr\linewidth-3mm}{
    \vspace{1mm}
    #1
    \vspace{1mm}
    }
    }%
}%
}

\newcommand{\imptext}[1]{%
\par%
\noindent\fcolorbox{black}{AGHgreen!7}{%
    \parbox{\dimexpr\linewidth-2\fboxsep-2\fboxrule}{
    \centering
    \parbox{\dimexpr\linewidth-3mm}{
    \vspace{1mm}
    #1
    \vspace{1mm}
    }
    }%
}%
}

\newcommand{\boxtext}[1]{%
	\begin{tcolorbox}[colback=gray!15,%gray background
		colframe=gray!15,% black frame colour
		width=\textwidth,% Use 5cm total width,
		arc=3mm, auto outer arc,
		]
		#1
	\end{tcolorbox}
}

\newcommand{\ety}[1]{_{\text{#1}}}
\newcommand{\upety}[1]{^{\text{#1}}}

\newcommand{\wektor}[1]{\left[\begin{array}{c}#1\end{array}\right]}

\newcommand{\der}{\mathrm{d}}

%–––––––––––––––––––––––––––––––––––––––––––––––––––––––
\newcommand{\multiline}[1]{\begin{tabular}{c}
		#1
\end{tabular}}
%–––––––––––––––––––––––––––––––––––––––––––––––––––––––
\newcommand{\doublepage}[2]{
	\noindent
	\begin{minipage}{.49\textwidth}
		#1
	\end{minipage}
	\begin{minipage}{.01\textwidth}
		\hfill
	\end{minipage}
	\begin{minipage}{.49\textwidth}
		#2
	\end{minipage}
	\vspace{1mm}
}
%–––––––––––––––––––––––––––––––––––––––––––––––––––––––




\title{\textsc{Introduction to spintronics - spin transistor}}
\author{Michał Modrzejewski}
\date{\today}

\begin{document}
	
\maketitle

\section*{Introduction}

One of characteristic properties of particles described in formalism of quantum mechanics is spin. Spin is important when we consider a behaviour of a particle in magnetic field. Manipulation of spin allow to build devices which in principle use a spin as a information carrier. This laboratory serves as a introduction to spintronic devices such as spin transistor.

\section*{Spin precession in magnetic field}

The first analysed system is a simple nano-wire with two contacts. Firstly a dispersion relation $E(k)$ without external magnetic field was calculated and can be seen below:

\begin{figure}[H]
	\centering
	\includegraphics[width=0.8\textwidth]{../plots/task1_disperion.pdf}
	\caption{Dispersion relation without external magnetic field.}
\end{figure}

Now that it's been confirmed, that a system is defined properly, a direction dependence of Zeeman splitting can be analysed. Below are presented dispersion relations for a nanowire with magnetic field defined for three spacial directions.

\begin{figure}[H]
	\begin{subfigure}{.33\textwidth}
		\includegraphics[width=\textwidth]{../plots/task1_disperion_x.pdf}
		\caption{$\mathbf{B} = [1,0,0]$ T}
	\end{subfigure}
	\begin{subfigure}{.33\textwidth}
		\includegraphics[width=\textwidth]{../plots/task1_disperion_y.pdf}
		\caption{$\mathbf{B} = [0,1,0]$ T}
	\end{subfigure}
	\begin{subfigure}{.33\textwidth}
		\includegraphics[width=\textwidth]{../plots/task1_disperion_z.pdf}
		\caption{$\mathbf{B} = [0,0,1]$ T}
	\end{subfigure}
	\caption{Dispersion relations for magnetic field applied in the directions $x,y$ or $z$.}
\end{figure}

As one can see the Zeeman splitting seems to be direction independent.

For a magnetic field defined as $\mathbf{B} = [0, 0, 1] $ T a conductance as a function of incident electron energy was calculated. Result are presented below:

\begin{figure}[H]
	\centering
	\includegraphics[width=0.8\textwidth]{../plots/task1_conductance.pdf}
	\caption{Conductance of a nanowire as a function of energy.}
\end{figure}

On a conductance function plot we can see a characteristic step-like behaviour on discreet quantized conductance values.

As a next step a spatially variant magnetic field was applied. Said magnetic field is defined so that for a whole nanowire applied magnetic field is $\mathbf{B} = [0,0,0.1]$ T. Additionally a $B_y$ component is added for region $x/L\in[0.2, 0.8]$, so effectively the magnetic field in this region is defined as $\mathbf{B} = [ 0, B_y , 0.1]$. For $B_y \in  [0,1]$ T spin dependent transmission coefficients were calculated. The results are presented in a figure below:

\begin{figure}[H]
	\centering
	\includegraphics[width=0.8\textwidth]{../plots/task1_transmissions_spin_dep.pdf}
	\caption{Spin dependent transmission coefficients as a $B_y$ function.}
\end{figure}

The transmission character changes periodically from conserved spin transition to spin flip transition. 

To further inspect the spin flip transition in a nanowire a spin dependent charge density in system was calculated. Since calculated transmission coefficient $T_{\text{up}\leftarrow\text{up}}$ for $B_y = 0.6$ T is about 0.92, that would mean we expect a transport with either no spin flip or spin flipping occurring so that the spin at both leads is the same. A calculated charge distribution can be seen below:

\begin{figure}[H]
	\centering
	\includegraphics[width=0.8\textwidth]{../plots/task1_charge_density.pdf}
	\caption{Spin dependent electron density in the nanowire.}
\end{figure}

The plot implies that during the transport spin "rotates" two times.

For the same system spin ($s_x,s_y,s_z$) density distribution was calculated and plotted on the picture below:

\begin{figure}[H]
	\includegraphics[width=\textwidth]{../plots/task1_spin_density.pdf}
	\caption{Spin distributions in the nanowire.}
\end{figure}

\section*{Spin transistor}

Including a spin-orbit interaction into calculations changes the outcomes and broadens the possibilities of effects to analyse. On a picture below a calculated dispersion relation with spin-orbit coupling inclusion is presented.

\begin{figure}[H]
	\centering
	\includegraphics[width=0.8\textwidth]{../plots/task2_disperion.pdf}
	\caption{Dispersion relation in the nanowire with SOC ($\alpha=50$ meVnm).}
\end{figure}

Including the spin-orbit coupling in calculations changes the shape of the dispersion relation. The bands are now horizontally split. 

The changes of conductance as a function of incident electron energy can also be analysed. The calculation results pictured on a figure can be seen below.

\begin{figure}[H]
	\centering
	\includegraphics[width=0.8\textwidth]{../plots/task2_conductance.pdf}
	\caption{Conductance as a function of electron energy with SOC considered.}
\end{figure}

Once again step-like behaviour can be observed, although now the steps take up two units of quantized conductance.

The spin dependent transmission coefficients were calculated as a function of $\alpha$ under assumption that the spin-orbit interaction is only present in the region $x/L\in[0.2, 0.8]$. The plotted results are visualized below.

\begin{figure}[H]
	\centering
	\includegraphics[width=0.8\textwidth]{../plots/task2_transmissions_spin_dep.pdf}
	\caption{Spin dependent transmission coefficients as a function of $\alpha$.}
\end{figure}

Analogically the spin dependent conductance can be calculated as a function of $\alpha$, the results are presented in a form of a plot below.


\begin{figure}[H]
	\centering
	\includegraphics[width=\textwidth]{../plots/task2_conds_spin_dep.pdf}
	\caption{Spin dependent conductance as a function of $\alpha$ for $P\in{0.1, 0.4, 1.0}$.}
\end{figure}

For a full polarization the total current is described by a single factor. For lower spin polarization of flowing current naturally lower total current is observed.

As the last task the density of spin up and spin down electrons and the spin $s_x, s_y,s_z$ density in the nanowire were calculated. Spin dependent electron densities maps are presented below.

\begin{figure}[H]
	\centering
	\includegraphics[width=0.8\textwidth]{../plots/task2_charge_density.pdf}
	\caption{Spin dependent electron densities.}
\end{figure}

Plot clearly shows how spin of a electron flips in the process of electronic transport in analysed system. As shown before, the spin dependent conductance changes periodically in function fo alpha, so it is possible that a corresponding outcome can be achieved by a numerous spin flip transitions. However in this case spin rotates only once.

The spin densities are plotted in a figure below.
\begin{figure}[H]
	\centering
	\includegraphics[width=\textwidth]{../plots/task2_spin_density.pdf}
	\caption{Spin densities in a nanowire, $\alpha=18$ meVnm.}
\end{figure}

\section*{Summary}

Nano-devices utilising spin as a additional degree of freedom can find many applications exceeding electronics for example in data storage. By tuning materials Rashba type spin-orbit interaction desired spin transport properties can be achieved, without the necessity for dependence on external magnetic fields. 



 
\end{document}
